%%%%%%%%%%%%%%%%%%%%%%%%%%%%%%%%%%%%%%%%%
% "ModernCV" CV and Cover Letter
% LaTeX Template
% Version 1.11 (19/6/14)
%
% This template has been downloaded from:
% http://www.LaTeXTemplates.com
%
% Original author:
% Xavier Danaux (xdanaux@gmail.com)
%
% License:
% CC BY-NC-SA 3.0 (http://creativecommons.org/licenses/by-nc-sa/3.0/)
%
% Important note:
% This template requires the moderncv.cls and .sty files to be in the same 
% directory as this .tex file. These files provide the resume style and themes 
% used for structuring the document.
%
%%%%%%%%%%%%%%%%%%%%%%%%%%%%%%%%%%%%%%%%%

%----------------------------------------------------------------------------------------
%	PACKAGES AND OTHER DOCUMENT CONFIGURATIONS
%----------------------------------------------------------------------------------------

\documentclass[11pt,a4paper,sans]{moderncv} % Font sizes: 10, 11, or 12; paper sizes: a4paper, letterpaper, a5paper, legalpaper, executivepaper or landscape; font families: sans or roman

\moderncvstyle{classic} % CV theme - options include: 'casual' (default), 'classic', 'oldstyle' and 'banking'
\moderncvcolor{grey} % CV color - options include: 'blue' (default), 'orange', 'green', 'red', 'purple', 'grey' and 'black'

\usepackage{lipsum} % Used for inserting dummy 'Lorem ipsum' text into the template
\usepackage{fontawesome}
\usepackage[scale=.85]{geometry} % Reduce document margins
\setlength{\hintscolumnwidth}{2.5cm} % Uncomment to change the width of the dates column
% \setlength{\makecvtitlenamewidth}{10cm} % For the 'classic' style, uncomment to adjust the width of the space allocated to your name

%----------------------------------------------------------------------------------------
%	NAME AND CONTACT INFORMATION SECTION
%----------------------------------------------------------------------------------------
\usepackage{tikz}

\usepackage{mathptmx}
\usepackage[11pt]{moresize}
\usepackage{xcolor}

\sectionfont{\color{cyan}}

\colorlet{H2Color}{teal!50!black}
\colorlet{L1Color}{teal!50!black}
\colorlet{L2Color}{teal!50!black}
\colorlet{E2Color}{black}
\colorlet{E3Color}{darkgray!50!gray}

\renewcommand{\section}[1]{
    \vspace*{1.0\baselineskip}
    \textcolor{H2Color}{{\Large #1}} \smallskip
    \textcolor{L1Color}{\hrule} \smallskip
    \textcolor{L1Color}{\hrule} \medskip
}


\newcommand{\SectionTight}[1]{
    \bigskip
    \textcolor{H2Color}{{\Large #1}} \smallskip
    \textcolor{L1Color}{\hrule} \smallskip
    \textcolor{L1Color}{\hrule} \medskip
}


\newcommand{\mycventry}[3]{
    \cventry{#1}{\mdseries \textcolor{E2Color}{#2}}{}{}{}{\textcolor{E3Color}{#3}} \smallskip      

}


\firstname{Sajjad} % Your first name
\familyname{Pakdamansavoji} % Your last name

% All information in this block is optional, comment out any lines you don't need


% \address{}{Tehran, Iran}
% \mobile{+98 - 9195919545}
% \email{sj.pakdaman.edu@gmail.com}
% \social[linkedin]{sajjad-pakdaman-savoji-9672221b6}
% \social[github]{sajjadpsavoji}





% \github{github.com/sajjadpsavoji} 

%%\extrainfo{
%%  \httplink{https://www.codechef.com/users/priyanka_pps}
%%\homepage{https://github.com/priyankapps/} {https://github.com/priyankapps/}}






\begin{document}
\begin{center}
    \Huge
    Sajjad Pakdamansavoji
\end{center}
\begin{center}
    \small
    \begin{tabular}{ c | c | c | c}
        \, \faMobile\enspace +1-647-835-6679 \, &  \, \href{mailto:savoji@yorku.ca}{\faEnvelopeO\enspace savoji@yorku.ca} \, & \,  \href{https://github.com/sajjadpsavoji}{\faGithub\enspace sajjadpsavoji} \, & \,  \href{https://www.linkedin.com/in/sajjad-pakdaman-savoji-9672221b6/}{\faLinkedin\enspace sajjad pakdaman savoji} \,  \\  
    \end{tabular}
\end{center}

% \makecvtitle 
%----------------------------------------------------------------------------------------
%	EDUCATION SECTION
%----------------------------------------------------------------------------------------

\SectionTight{EDUCATION}
\mycventry{2021--2023}{MSc in Computer Science, York University}{GPA: 8.66/9 \qquad Supervised by Professor \href{https://www.elderlab.yorku.ca/jelder/}{James Elder}}
\mycventry{2016--2021}{BSc in Electrical Engineering, University of Tehran}{GPA: 3.77/4 \qquad Supervised by Assoc.Prof. \href{https://ece.ut.ac.ir/en/~akalhor}{Ahmad Kalhor}}
\mycventry{2018--2020}{Minor Degree in Computer Engineering, University of Tehran}{GPA: 18.2/20\qquad Ranked $2^{nd}$}
% \mycventry{2012--2016}{Diploma in Mathematics and Physics, NODET Allameh Helli 8 Branch}{National Organization for Development of Exceptional Talents \\ GPA: 19.73/20}


%----------------------------------------------------------------------------------------
%	AWARDS SECTION
%------------   ----------------------------------------------------------------------------
\section{AWARDS AND ACHIEVEMENTS}
{
    \cvitem{Sep 2022}{VISTA Graduate Scholarship, Vision: Science to Application (2 times)}
    \cvitem{Apr 2021}{Vector Scholarship in AI, Vector Institute}
    \cvitem{Jun 2021}{Lassonde Entrance Scholarship, York University}
    % \cvitem{May 2021}{VISTA Graduate Scholarship, Vision: Science to Application}
\cvitem{Jan 2018}{Supporter Foundation of University of Tehran Scholorship}
% \cvitem{Oct 2017}{Ranked $93^{th}$ worldwide in IEEEXtreme 11.0 amongst 2121 teams}
% \cvitem{Nov 2017}{Silver medal in basketball competitions, $16^{th}$ student sport festival, University of Tehran}
% \cvitem{Nov 2016}{Gold medal in basketball competitions, $15^{th}$ student sport festival, University of Tehran}
% \cvitem{Jul 2016}{Ranked 600th in Iranian University Entrance Exam among 162879 participants(top 0.36\%)}
% \cvitem{Nov 2011}{Gold medal in basketball competitions, city of Tehran's inter-secondary school sport festival}
\par}
% \mycventry{Nov 2017}{Silver medal in basketball competitions, $16^{th}$ student sport festival, University of Tehran}{}
% \mycventry{Nov 2016}{Gold medal in basketball competitions, $15^{th}$ student sport festival, University of Tehran}{}
% \mycventry{Jul 2016}{Placed among top $0.36 \%$ high school students in Iran National University Entrance Exam}{}
% \mycventry{Nov 2011}{Gold medal in basketball competitions, city of Tehran's inter-secondary school sport festival}{}

% \SectionTight{INTERESTS}
% {\small
% Artificial Intelligence, Deep Learning, Machine Learning, Computer Vision, Natural Language Processing, Generative Models
% }

%----------------------------------------------------------------------------------------
%	Experience 
%----------------------------------------------------------------------------------------

\SectionTight{WORK EXPERIENCE}
% \mycventry{Jun'21--Jun'22}{Committee Member, CUPE Toronto District Council}{Elected as the CUPE local 3903 representative in this council}
% \mycventry{Jul'19--Sep'20}{Strategic Advisor, IEEE University of Tehran Student Branch}{Provided branch's executive committee with decisive advice and support}
\mycventry{Apr'20--Jun'20}{Data Science Mentor, Amirkabir University of Technology}
{
    - Prepared course materials on python, ML,and Data Sience\\ 
    - Held tutorial sessions and QA meetings
}
\mycventry{Nov'19--Jan'20}{Machine Learning Mentor, IEEE UTSB Data Science Winter School}
{
    - Held hands-on sessions on how to use sci-kit learn library\\
    - Held hands-on session on sci-py and statistical inference
}
\mycventry{Jul'19--Sep'19}{Deep Learning Intern, AVIR Company}
{
   - Implemeted hand detection and gesture prediction using YOLO and DeepSort
}
\mycventry{Jul'18--Jul'19}{Vice chair, IEEE UT Student Branch}{- Organized the student branch's different sections and managed the exec team}
% \mycventry{Jul'18--Aug'18}{IoT Academy $4^{th}$ Summer School, University of Tehran Science \& Technology Park}{With subject of "Internet of Things Rapid Prototyping from A to Z"}
\section{RESEARCH EXPERIENCE}
\mycventry{Sep'21--Aug'23}{RA, Center of Vision Research (CVR), York University}
    {
        Leveraged SOTA multiple object detection and tracking on private surveillance footage to \\
        - Localize the groundplane location of vehicles using calibrated cameras \\
        - Improved turn-counts estimation error from 20\% to 3\%
    }
\mycventry{Sep'20--Jul'21}{RA, Machine Learning Lab, University of Tehran}
    {
        Manifold learning on the CIFAR10 dataset using ResNet50 backbone and contrastive loss\\
        Per-layer feature seprability analysis of ResNet using Seperation Index(our novel metric)
        % supervisor: \href{https://ece.ut.ac.ir/en/~akalhor}{Ahmad Kalhor}, Assoc.Prof.  
    }
\mycventry{Sep'19--Sep'20}{RA, Computer Networks Lab, University of Tehran}
    {
        Predicted future packet-type for each user in a local network\\
        - Classified packet-type using deep packet inspection (DPI) tools\\
        - Forcasted user behaviour using recurrent neural networks(RNN) and LSTMs
        % supervisor: \href{https://ece.ut.ac.ir/en/~vmansouri}{Vahid Shahmansouri}, Asst.Prof.  
    }
\mycventry{Jul'19--Sep'19}{Research Intern, Secure Communication Lab, University of Tehran}
    {
        Iranian Sign Language (ISL) translation from visual footage to text\\
        - Extracted key-point features from hand and face gestures \\
        - Translated word-level feature sequences using RNNs and LSTMs
        % supervisor: \href{https://ece.ut.ac.ir/en/~akhaee}{Mohammad Ali Akhaee},  Asst.Prof. 
    }



%----------------------------------------------------------------------------------------
%	Skills SECTION
%----------------------------------------------------------------------------------------
\SectionTight{TECHNICAL SKILLS}
\renewcommand{\listitemsymbol}{-~}
{\small
\cvitem{Languages}{Python{\tiny(advance)}, C++{\tiny(advance)}, C{\tiny(intermediate)}}
\cvitem{Tools}{Tensorflow, Keras, Pytorch, Scikit-learn, Numpy, Pandas, OpenCV, Jupyter, MATLAB, JAX}
% \cvitem{Softwares}{MATLAB, \LaTeX, Office programs, Quartus, NI Multisim, Codevision, Proteus}
% \cvitem{Digital Devices}{AVR ATmeg series, Raspberry Pi3, Arduino}
\cvitem{OS.s}{Linux, Windows, macOS}}


%----------------------------------------------------------------------------------------
%	Masters SECTION
%----------------------------------------------------------------------------------------

%\section{Masters Thesis}

%\cvitem{Title}{\emph{Technologies and characterization of ferroelectric polymers for biomedical sensors}}
%\cvitem{Supervisors}{Professor Antonino Fiorillo}
%\cvitem{Description}{This thesis is based on the implementation of a temperature sensor.}

%----------------------------------------------------------------------------------------
%	WORK EXPERIENCE SECTION
%----------------------------------------------------------------------------------------
\SectionTight{PROJECTS}
% \mycventry{October'19}{\href{https://github.com/SajjadPSavoji/analog_modulatio_am_dsb_ssb}{Analog Communication Simulator}}{Implemented transmitter and receiver for several analog modulations such as AM, DSB, LSSB, USSB, PM and FM}
% \mycventry{August'20}{\href{https://github.com/SajjadPSavoji/sdn_routing}{Shortest-Path Routing in Software Defined Networks}}{Implemented Dijkstra algorithm using Ryu controller and Mininet}
% \mycventry{July'20}{\href{https://github.com/SajjadPSavoji/gnutella_p2p}{Gnutella}}{From scratch implementation of a peer-to-peer network scheme inspired by Gnutella}
% \mycventry{June'20}{\href{https://github.com/SajjadPSavoji/tcp-congestion-control-in-ns2}{TCP Congestion Control}}{Illustrated congestion control mechanism using NS2}
% \mycventry{May'20}{\href{https://github.com/SajjadPSavoji/FTP_Server}{File Transfer Protocol Server}}{From scratch Object Oriented based implementation of a FTP server}
\mycventry{Jun'22}{\href{}{Traffic Flow Analysis at Intersections}}{
    - remove non-linear distortion from wide-angle surveillance camera\\
    - perform multi-object detection and multi-object tracking on image plane\\
    - back-project tracks into ground plane and extract data such as turn counts
}
\mycventry{Dec'21}{\href{}{Celebrity Face Synthesis}}{
    - trained Real-Valued Non-Volume Preserving Normalizing Flow to generate faces\\
    - trained Deep Convolotional Generative Adversarial Networks to generate faces\\
    - trained Variational Auto Encoders with Convolutional layers to generate faces
}

\mycventry{Nov'21}{\href{}{Sentiment Analysis of twits using Neural Language Models}}{
    - used N-gram and Bag-Of-Words models for hate speech detection\\
    - used Bidirectional Encoder Representations from Transformers(BERT) for hate speech detection\\
    - used Generalized Autoregressive Pretraining (XLnet) for hate speech detection
}

\mycventry{July'21}{\href{}{Machine Translation Using Transformers}}{applying transformer-based models(BERT) for English to Persion translation}
\mycventry{Feb'21}{\href{}{Sentence Generation}}{applying LSTM-based Language model for auto-regressive sentence generation}
\mycventry{Nov'20}{\href{}{Generative Models for Natural Scene Synthesis}}{
    - trained Deep Convolutional GAN (DCGAN) for image generation from CFAR10\\
    - trained Deep Conditional GAN (CGAN) for image generation from CFAR10\\
    - trained Deep Auxiliary Classifier GAN (AC-GAN) for image generation from CFAR10
}
% \mycventry{April'21}{\href{}{Sentiment Analysis with BERT and XLNet}}{Used transformer-based embedding to detect hate speech}
% \mycventry{Murch'21}{\href{}{Sentence Classification with Ngrams}}{Used perplexity of an interpolated unigram-bigram language model for classification}
% \mycventry{June'20}{\href{https://github.com/SajjadPSavoji/Twitter-Sentiment-Analysis-HMM}{Twitter Sentiment Analysis}}{Used Hidden Markov Model to formulate a classifier on twits}
% \mycventry{September'20}{\href{https://github.com/SajjadPSavoji/Auxiliary-Classifier-GAN}{Auxiliary Classifier Generative Adversarial Network (ACGAN)}}{Implemented an ACGAN using Keras on CIFAR10 dataset}
% \mycventry{August'20}{\href{https://github.com/SajjadPSavoji/Conditional-GAN}{Conditional Generative Adversarial Network (CGAN)}}{Implemented a CGAN using Keras on CIFAR10 dataset}
% \mycventry{December'19}{\href{https://github.com/SajjadPSavoji/nlp_naive_bayes}{Poem Classification}}{Used NLP methods such as bag-of-words to develop a bayes classifier for poetry}
% \mycventry{July'20}{\href{https://github.com/SajjadPSavoji/Deep-Convolutional-GAN}{Deep Convolutional Generative Adversarial Network (DCGAN)}}{Implemented a DCGAN using Keras on CIFAR10 dataset}
% \mycventry{June'20}{\href{https://github.com/SajjadPSavoji/Variational-Auto-Encoder}{Variational Auto Encoder (VAE)}}{Used Kullback-Leibler Divergance cost to tarin a VAE on MNIST dataset}
% \mycventry{June'20}{\href{https://github.com/SajjadPSavoji/Digital_Communication_Lab}{Digital Communication Simulator}}{Implemented transmitter and receiver for several digital modulations such as pam, psk, fsk, etc all of which support differentiation coding and pulse-shape selection}
% \mycventry{May'20}{\href{https://github.com/SajjadPSavoji/Pollution-Predicttion}{Pollution Prediction}}{Implemented several recurrent networks using different cells such as LSTM and GRU}
\mycventry{May'20}{\href{https://github.com/SajjadPSavoji/German-Traffic-Sign-Recognition-Benchmark}{German Traffic Sign Recognition Benchmark}}{Used deep convolutional NNs to classify traffic signs used in germany}
% \mycventry{July'20}{\href{https://github.com/SajjadPSavoji/DBSCAN-vs-Kmeans}{DBSCAN vs Kmeans}}{A rough comparison between two clustering algorithms and the advantages DBSCAN provides}
% \mycventry{July'20}{\href{https://github.com/SajjadPSavoji/bbox-clustering}{Boundary Box clustering}}{Performed Kmeans clustering on PASCAL VOC dataset with IOU distance}
% \mycventry{July'20}{\href{https://github.com/SajjadPSavoji/Image-Compression-Kmeans}{Image Compression}}{Used Kmeans clustering to substitute centroids with corresponding pixels}
% \mycventry{April'20}{\href{https://github.com/SajjadPSavoji/Mechanism-Based-Neural-Nerworks}{Mechanism Based Neural Networks}}{Implemented some basic mechanism based NNs like SOM, MaxNet, MexicanHat and Hamming Net}
% \mycventry{April'20}{\href{https://github.com/SajjadPSavoji/Memory-Associative-Networks}{Memory Associative Neural Networks}}{From scratch implementation of Auto-associative, Iterative Auto-associative and Hetro-associative nets}
\mycventry{Murch'20}{\href{https://github.com/SajjadPSavoji/SeperationIndex_for_CNN}{Separation Index trends in Fully Convolutional Networks}}{Computed published separation metrics to examine feature flow through layers of CNNs}
% \mycventry{March'20}{\href{https://github.com/SajjadPSavoji/Madeline_MLP}{Many Adaptive Linear Neuron (MADALINE)}}{From scratch implementation of ADALINE and MADALINE}
% \mycventry{February'20}{\href{https://github.com/SajjadPSavoji/Character_Classifer_MLP}{Multi Layer Perceptron (MLP)}}{From scratch implementation of MLP to classify written characters}
% \mycventry{September'19}{\href{https://github.com/SajjadPSavoji/American_Sign_Language}{American Sign Language Translation}}{Combined CNNs and RNNs alongside to develop an alphabet-level language translator}
% \mycventry{August'19}{\href{https://github.com/SajjadPSavoji/Object_Localization}{Localize Fish Instances on the Boat}}{implemented YOLO2 network to localize fishes while specifying their breeds}
% \mycventry{July'19}{\href{https://github.com/SajjadPSavoji/HumpbackWhale_Identification}{Humpback Whale Identification}}{Used CNNs to develope an identification system based-on whales' tails}
% \mycventry{June'19}{\href{https://github.com/SajjadPSavoji/speech_identification}{Speech Identification in MEL Spectrum}}{studied the effect of MEL transformation on speech identification with NNs}
% \mycventry{June'19}{\href{https://github.com/SajjadPSavoji/FaceRecognition}{Face Recognition}}{Used the siamese network alongside with triple-loss cost function on AT\&T dataset}
% \mycventry{October'19}{\href{https://github.com/SajjadPSavoji/SVM}{Retail Classification}}{A kernelized support vector machine was used to address the problem}
% \mycventry{October'19}{\href{https://github.com/SajjadPSavoji/PCA_LDA}{Principal component analysis (PCA)}}{From scratch implementation of PCA scheme for dimension reduction}
% \mycventry{October'19}{\href{https://github.com/SajjadPSavoji/PCA_LDA}{Linear discriminant analysis (LDA)}}{From scratch implementation of LDA scheme for dimension reduction}
% \mycventry{November'19}{\href{https://github.com/SajjadPSavoji/parzen_window}{Parzen window}}{One of the major non-parametric pdf estimation methods used in a bayes optimal classifier}
% \mycventry{November'19}{\href{https://github.com/SajjadPSavoji/random_forest}{Random Forest}}{Used boosting and bagging to develope Random Forest model from predefined Decision Tree classifier }
% \mycventry{December'19}{\href{https://github.com/SajjadPSavoji/minimax}{MiniMax}}{Implemented the minimax adversarial approach for a sample zero-mean game}
% \mycventry{December'19}{\href{https://github.com/SajjadPSavoji/pacman}{Pacman}}{Used heuristic based agents as the back logic for Pacman game}


% \mycventry{September'18}{\href{https://github.com/SajjadPSavoji/AP_Drive}{AP Drive}}{A File Hosting Service inspired by Drop Box using C++ with a web based GUI}
% \mycventry{October'18}{\href{https://github.com/SajjadPSavoji/Mafia}{Mafia Game}}{A CLI based game designed according to OO paradigm using C++}
% \mycventry{November'18}{\href{https://github.com/SajjadPSavoji/OnlineStore}{Online Store}}{Object Oriented modeling a virtual store to manage trades and transactions of goods using C++}
% \mycventry{December'18}{\href{https://github.com/SajjadPSavoji/kingdomrush}{Kingdom Rush}}{A tower-defense game implemented by Simple DirectMedia Layer(SDL) graphical library using C++}

% \mycventry{September'19}{\href{https://github.com/SajjadPSavoji/synchronization_using_mutex_locks}{Synchronization by Mutex Lock}}{An object oriented implementation to synchronize multiple processes}
% \mycventry{October'19}{\href{https://github.com/SajjadPSavoji/multiprocess_ensemble_learning}{Multiprocess ensemble learning}}{Used multiprocessing to implement ensemble linear classifiers}
% \mycventry{November'19}{\href{https://github.com/SajjadPSavoji/xv6}{XV6}}{Added several features to the original XV6 operating system designed in MIT, such as paging, reentrance locks, new system-calls and new scheduling schemes}
% \mycventry{December'19}{\href{https://github.com/SajjadPSavoji/client-server_and_P2P_communication-}{Hybrid File Sharing}}{Implemented a file sharing application that supports both client-server and peer-2-peer methods}


%----------------------------------------------------------------------------------------
%	Teaching Experience section
%----------------------------------------------------------------------------------------
\section{TEACHING ASSISTANT EXPERIENCES}
\mycventry{Sep'22--Dec'22}{Design and Analysis of algorithms, York University}
{
    - held in-person tutorials and graded assignments 
    % \emph{Instructor:} \href{https://scholar.google.com/citations?user=hQXlVLsAAAAJ}{\underline{Shahin Kamali}, Assist.Prof, York University}
}
\mycventry{Sep'20--Dec'22\\\textbf{\footnotesize graduate course}}{Neural Networks and Deep Learning (4 semesters), University of Tehran}
{
    - designed the final project regarding Generative Adversarial Networks
    % \emph{Instructor:} \href{https://scholar.google.com/citations?hl=en&user=m7xdmMgAAAAJ}{\underline{Ahmad Kalhor}, Assoc.Prof, University of Tehran}
}
\mycventry{Jan'22--Apr'22\\\textbf{\footnotesize graduate course}}{Data Mining, York University}
{
    - graded projects, assignment, and exams
    % \emph{Instructor:} \href{https://www.researchgate.net/profile/Habib-Ur-Rehman-14}{\underline{Habib-ur Rehman}, Adjunct Prof, McMaster University}
}
\mycventry{Sep'21--Nov'21}{Software Tools, York University}
{
    - held hands-on sessions and grading projects
    % \emph{Instructor:} \href{https://ca.linkedin.com/in/huiwang0123}{\underline{Hui Wang}, Reasearch Assoc, York University}
}
\mycventry{Sep'20--Feb'21\\\textbf{\footnotesize graduate course}}{Machine Learning (2 semesters), University of Tehran}
{
    - held hands-on session, designed assignments
    % \emph{Instructor:} \href{https://scholar.google.com/citations?hl=en&user=FTcata0AAAAJ}{\underline{Babak Nadjar Araabi}, Prof, University of Tehran}
}
% \mycventry{Feb'20--Jun'20}{Teaching assistant for Pattern Recognition \textbf{(graduate-level course)}}{Held hands-on session, designed 2 homeworks and marked them, designed 3 quizzes and marked them, graded their final project, suggested questions for midterm and final exam}
% \mycventry{Feb'20--Jun'20\\\textbf{\footnotesize graduate course}}{Pattern Recognition, University of Tehran}
% {
%     - held hands-on session, designed 2 homeworks and 3 quizzes
%     % \emph{Instructor:} \href{https://ece.ut.ac.ir/en/~dehaqani}{\underline{Mohammadreza Abolghasemi}, Asst.Prof,  University of Tehran}
% }
\mycventry{Sep'20--Feb'21}{Intelligent systems, University of Tehran}
{
    - designed 6 final projects and marked them, individual assessment
    % \emph{Instructor:} \href{https://scholar.google.com/citations?user=zqa4EY0AAAAJ&hl=en}{\underline{Reshad Hoseini}, Asst.Prof, University of Tehran}
}
% \mycventry{Feb'20--Feb'21\\\textbf{\footnotesize graduate course}}{Digital Signal Processing (3 semesters), University of Tehran}
% {
%     - designed 4 CAs, one analytical assignment and organized other TAs
%     % \emph{Instructor:} \href{https://scholar.google.com/citations?hl=en&user=Yqzkf3UAAAAJ}{\underline{Hadi Amiri}, Asst.Prof, University of Tehran}\\
%     % \emph{Instructor:} \href{https://scholar.google.com/citations?user=VspDmN8AAAAJ&hl=en}{\underline{Majid Badieirostami}, Asst.Prof, University of Tehran}\\
%     % \emph{Instructor:} \href{https://scholar.google.com/citations?user=atXIykYAAAAJ&hl=en}{\underline{Mohammadali Akhaee}, Asst.Prof, University of Tehran}
% }
% \mycventry{Feb'20--Jun'20}{Teaching assistant for Digital Signal Processing}{Desinged 2 computer assignments and a homework}
% \mycventry{Sep'19--Feb'21}{TA, Communication Systems I (3 semesters), University of Tehran}
% {
%     \emph{Obligation:} designed 4 homeworks all of which include implementation part and 4 CAs, assessed students
%     % \emph{Instructor:} \href{https://scholar.google.com/citations?user=fCts8c0AAAAJ&hl=en}{\underline{Maryam Sabaghian}, Asst.Prof, University of Tehran}\\
%     % \emph{Instructor:} \href{https://scholar.google.com/citations?hl=en&user=Yqzkf3UAAAAJ}{\underline{Hadi Amiri}, Asst.Prof, University of Tehran}
% }

% \mycventry{Feb'20--Jun'20}{Teaching assistant for Communication Systems I}{Desined a computer assignment, assesed their performance via interview sessions}
% \mycventry{Sep'19--Feb'20}{Teaching assistant for Communication Systems I}{Desined 3 computer assignments, assesed their programming skills individually}

\mycventry{Feb'19--Feb'20}{Engineering Probability and Statistics (2 semesters), University of Tehran}
{
    - designed homeworks and computer assignments, held Q\&A session
    % \emph{Instructor:} \href{https://ece.ut.ac.ir/en/~dehaqani}{\underline{Mohammadreza Abolghasemi}, Asst.Prof, University of Tehran}
}
% \mycventry{Feb'19--Jun'19}{Teaching assistant for Engineering Probability and Statistics}{Designed homeworks, held Q\&A session, marked their homeworks}
% \mycventry{Feb'19--Jun'19}{TA, Electronics I}
% {
%     \emph{Obligation:} marked student's homeworks, mentored them in the problem-solving procedure\\
%     \emph{Instructor:} \href{https://scholar.google.com/citations?hl=en&user=T7fVw08AAAAJ}{\underline{Mohammadreza Kolahdooz}, Asst.Prof, University of Tehran}
% }






%------------------------------------------------
%----------------------------------------------------------------------------------------
%	certification and workshops
%----------------------------------------------------------------------------------------
% \section{CERTIFICATES}
% {\small
% \renewcommand{\listitemsymbol}{-~} % Changes the symbol used for lists

% \cvitem{January 2022}{\href{https://www.coursera.org/account/accomplishments/certificate/XC3RQE34YAXS}{Build Better Generative Adversarial Networks (GANs), Coursera Credential: XC3RQE34YAXS}}
% \cvitem{January 2022}{\href{https://www.coursera.org/account/accomplishments/certificate/J4YEGV3VWSQ3}{Sample-based Learning Methods, Coursera Credential: J4YEGV3VWSQ3}}
% \cvitem{Novermber 2021}{\href{https://www.coursera.org/account/accomplishments/certificate/B9M4EX5HRXG3}{Build Basic Generative Adversarial Networks (GANs), Coursera Credential: B9M4EX5HRXG3}}
% \cvitem{October 2021}{\href{https://www.coursera.org/account/accomplishments/certificate/DMPL7YR2KURX}{Fundamentals of Reinforcement Learning, Coursera Credential: DMPL7YR2KURX}}
% \cvitem{Oct 18 - Oct 20}{Ambassador for IEEEXtreme 12.0, 13.0, and 14.0 Programming Competition}
% \cvitem{January 2020}{Contributed as a Statistical Inference Instructor in IEEEUTSB Data Science Winter School}
% \cvitem{February 2019}{Participated in the international workshop on "signal proccessing" held in Sharif University of Technology}
% \cvitem{April 2019}{Ambassador for IEEEmadC Competition}
% \cvitem{October 2019}{Contibuted as an Ambassador in IEEEXtreme 13.0 Programming Competition}
% \cvitem{August 2018}{Attended IoT Academy $4^{th}$ Summer School}
% \cvitem{October 2018}{Contibuted as an Ambassador in IEEEXtreme 12.0 Programming Competition}
% \cvitem{September 2017}{Attended MATLAB programming course, held by IEEEUTSB}
% \cvitem{May 2017}{Attended AVR Microcontroller course, held by IEEEUTSB}
% \par}

% \section{Volunteering}
% {\small
% \renewcommand{\listitemsymbol}{-~} % Changes the symbol used for lists
% \cvitem{Dec 21 - Dec 22}{Student representative, Vector Institute for Artificial Intelligence}
% \cvitem{Apr 19 - Apr 20}{Ambassador for IEEEmadC Competition}
% \cvitem{Oct 18 - Oct 20}{Ambassador for IEEEXtreme Programming Competition (12.0, 13.0, and 14.0)}
% \par}


%\section{Major Courses Undertaken}

%\renewcommand{\listitemsymbol}{-~} % Changes the symbol used for lists
%{\small
%\cvitem{Computer Science}{ Computer Graphics and Multimedia, Artificial Intelligence, Design and Analysis of Algorithms, Theory of Computation, Operating Systems, Programming Languages, Database Management System, Computer Architecture, Discrete Mathematics, Data Structures.
%}

%\cvitem{Mathematics}{ Statistical Methods, Probability, Differential Calculus, Linear Algebra, Matrix Theory, Complex Analysis, Vector Calculus.
%}
%\par}

%----------------------------------------------------------------------------------------
%	Coursework SECTION
%----------------------------------------------------------------------------------------


%----------------------------------------------------------------------------------------
%	COMMUNICATION SKILLS SECTION
%----------------------------------------------------------------------------------------

    %\section{Communication Skills}

    %\cvitem{2010}{Oral Presentation at the California Business Conference}
    %\cvitem{2009}{Poster at the Annual Business Conference in Oregon}

    %----------------------------------------------------------------------------------------
    %	LANGUAGES SECTION
    %----------------------------------------------------------------------------------------
%     \section{LANGUAGES}

%     \cvitemwithcomment{English}{Professional Profiency {\footnotesize{(IELTS 8.0)}}}{}
% \cvitemwithcomment{Persian/Farsi}{Native}{}
%----------------------------------------------------------------------------------------
%	INTERESTS SECTION
%----------------------------------------------------------------------------------------
%  \section{REFERENCES}
%     \begin{minipage}[t]{0.33 \textwidth}
%         \footnotesize
%         \href{https://ece.ut.ac.ir/en/~akalhor}{\textbf{Ahmad Kalhor}
%         }\\
%         Assoc. Professor\\
%         University of Tehran\\
%         Deep Learning\\
%         \href{mailto:akalhor@ut.ac.ir}{{\tiny \faEnvelopeO} akalhor@ut.ac.ir}
%     \end{minipage}
%     \begin{minipage}[t]{0.33 \textwidth}
%         \footnotesize
%         \href{https://ece.ut.ac.ir/en/~reshad.hosseini}{\textbf{reshad hosseini}
%         }\\
%         Asst. Professor\\
%         University of Tehran\\
%         Machine Intelligence and Robotics\\
%         \href{mailto:reshad.hosseini@ut.ac.ir}{{\tiny \faEnvelopeO} reshad.hosseini@ut.ac.ir}
%     \end{minipage}
%     \begin{minipage}[t]{0.33 \textwidth}
%         \footnotesize
%         \href{https://ece.ut.ac.ir/en/~dehaqani}{\textbf{Mohammadreza Abolghasemi}
%         }\\
%         Asst. Professor\\
%         University of Tehran\\ 
%         Machine Intelligence and Robotics\\
%         \href{mailto:dehaqani@ut.ac.ir}{{\tiny \faEnvelopeO} dehaqani@ut.ac.ir}
%     \end{minipage}


% \section{Interests}

% \renewcommand{\listitemsymbol}{-~} % Changes the symbol used for lists
% \cvlistdoubleitem{Piano}{}
% \cvlistitem{Baseball}

%----------------------------------------------------------------------------------------
%	COVER LETTER
%----------------------------------------------------------------------------------------

% To remove the cover letter, comment out this entire block

%\clearpage

%\recipient{HR Department}{Corporation\\123 Pleasant Lane\\12345 City, State} % Letter recipient
%\date{\today} % Letter date
%\opening{Dear Sir or Madam,} % Opening greeting
%\closing{Sincerely yours,} % Closing phrase
%\enclosure[Attached]{curriculum vit\ae{}} % List of enclosed documents

%\makelettertitle % Print letter title

%\lipsum[1-3] % Dummy text

%\makeletterclosing % Print letter signature

%----------------------------------------------------------------------------------------

\end{document}



